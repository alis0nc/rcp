\documentclass{article}
\usepackage{cite}
\usepackage{hyperref}
\usepackage{xcolor}
\hypersetup{
    colorlinks,
    linkcolor={red!50!black},
    citecolor={blue!50!black},
    urlcolor={blue!80!black}
}
\author{alisonc}
\title{Notes for literature review}
\date{2015-04-23}

\begin{document}
\maketitle

Lots of design decisions for MPTCP were made with the goal of making it 
compatible with Plain Old TCP and the myriad of middleboxes in the today's 
internet. This is not something we need to be concerned about. \cite{rfc6824}

MPTCP: the same data can be sent on multiple subflows for resilience. The first 
copy that arrives should be taken as authoritative and delivered to the application;
further are ignored. \cite[p.~25]{rfc6824}

Retransmissions are ``clearly suboptimal''. All retransmissions on a different subflow 
first require retransmission on the original subflow. Why? Because compatibility with
legacy middleboxes. \cite[\S~3.3.6]{rfc6824} Also, much about subflows depends on local policy. 
\cite[\S~3.3.8]{rfc6824}

``Multiple retransmissions are triggers that will indicate that a
subflow performs badly and could lead to a host resetting the subflow
with a RST.  However, additional research is required to understand
the heuristics of how and when to reset underperforming subflows.
For example, a highly asymmetric path may be misdiagnosed as
underperforming.'' \cite[p.~33]{rfc6824}

Subflow priority is binary -- either main or backup. This is just a suggestion and
not binding though.

Address adding and removing via \texttt{ADD\_ADDR} and \texttt{REMOVE\_ADDR} depends 
on the hosts view of its network connections, not the topology. \cite[\S~3.4]{rfc6824}.

MPTCP has some shortcomings:
\begin{itemize}
  \item No network level view of topology. can't pick the $k$ shortest paths
  \item Most applicable to multihomed systems
  \item data redundancy is implement badly
  \item All the design compromises to be compatible with legacy middleboxen
\end{itemize}

\bibliographystyle{plain}
\bibliography{sources.bib}
\end{document}